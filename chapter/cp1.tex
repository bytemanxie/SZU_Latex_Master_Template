\clearpage

\section{绪论}

\subsection{研究背景及意义}

随着现代工业制造向精密化、自动化方向发展,激光焊接技术凭借能量密度高、焊接速度快、热影响区小等特点,在汽车制造与动力电池等高端装备制造领域得到了广泛应用\upcite{HuangYiwei_OCT_WeldDepth}。在叠焊等典型工艺中,熔深直接影响接头的有效连接与承载能力,是衡量焊接质量的关键指标之一\upcite{HuangYiwei_OCT_WeldDepth}。然而,激光焊接过程涉及光、机、电、热等多物理场耦合,具有明显的非线性与不稳定性,易产生孔隙、未熔合等缺陷;若仍依赖金相切片等焊后离线检测,往往存在破坏性强、效率低、反馈滞后等问题,难以满足规模化生产的实时质检与闭环控制需求\upcite{HuangYiwei_OCT_WeldDepth}。因此,面向焊接过程的实时在线监测与熔深稳定获取具有重要的工程价值。

光学相干层析成像(Optical Coherence Tomography, OCT)是一种基于低相干干涉原理的层析成像技术,具有非接触、高轴向分辨率等优势\upcite{ZengHaitao_OCT_Speckle}。在激光焊接场景中,OCT 可通过测量光束获取匙孔(Keyhole)底部反射信息,从而为熔深在线测量提供直接的几何依据,被认为是具有应用潜力的过程监测手段之一\upcite{HuangYiwei_OCT_WeldDepth}。相较于仅能反映表面辐射或二维表观信息的视觉/光电监测方法,OCT 在表征深度结构方面更具优势\upcite{HuangYiwei_OCT_WeldDepth}。

尽管 OCT 在焊接监测中具有优势,但在实际应用中仍面临信号不稳定与噪声干扰等挑战:焊接过程中的熔池波动、金属蒸汽与飞溅会降低信号稳定性\upcite{HuangYiwei_OCT_WeldDepth};同时,OCT 成像基于低相干干涉,散斑噪声(Speckle Noise)常以随机颗粒纹理形式出现并降低图像信噪比\upcite{ZengHaitao_OCT_Speckle}。在此条件下,锁孔边缘往往呈现对比度低、边界模糊与形态快速变化等特点,传统阈值分割、边缘检测或简单滤波(如中值滤波、百分位滤波)难以在鲁棒性与边界保持之间取得平衡,进而影响后续熔深估计的稳定性与精度\upcite{HuangYiwei_OCT_WeldDepth,ZengHaitao_OCT_Speckle}。

针对上述痛点,本文研究面向焊接 OCT 图像的语义分割方法。与传统图像处理流程相比,深度学习分割模型可通过端到端学习获取更具判别性的语义特征与上下文信息,在噪声与弱边界条件下具有更强的鲁棒性\upcite{Long2015FCN,Ronneberger2015UNet,Chen2018DeepLab}。本文以 DeepLabV3+ 为基线\upcite{Chen2018DeepLab},面向散斑噪声干扰强、锁孔边界模糊与细长结构易断裂等问题,在编码器端增强全局上下文建模,在解码器端强化局部细节表达,从而实现对锁孔区域的高精度分割,为熔深稳定测量提供算法支撑\upcite{HuangYiwei_OCT_WeldDepth}。

\subsection{国内外研究现状}

\noindent 本节围绕本文研究对象(激光焊接场景下的 OCT 锁孔/熔深信息获取)与研究任务(OCT 图像中锁孔目标的语义分割)展开综述。首先总结激光焊接过程监测与熔深测量的主流技术路线及其局限性;其次梳理 OCT 成像中的散斑噪声特性与典型处理方法,指出“先去噪再分割/测量”的流程在边界保持与实时性方面的不足;最后回顾图像语义分割算法的发展脉络与在 OCT/工业视觉中的应用进展,从而引出本文面向噪声干扰与弱边界条件的端到端抗噪分割思路。

\subsubsection{激光焊接过程监测技术现状}
\noindent 激光焊接过程监测手段大体可分为基于光辐射/光谱的光电监测、基于可见光/红外成像的视觉监测、基于声发射/等离子体信号的声学监测,以及融合多源传感的综合监测等。上述方法在一定程度上能够反映焊接稳定性与缺陷趋势,但往往难以直接、稳定地获取锁孔内部几何信息;同时其信号与熔深之间的映射关系受工艺参数、材料状态与飞溅/蒸汽干扰影响显著,导致可迁移性与精度受限。在此背景下,OCT 凭借其高轴向分辨率、非接触、可穿透烟尘与金属蒸汽并直接表征锁孔深度结构等优势,被认为是实现熔深在线测量与闭环控制的有前景手段\upcite{HuangYiwei_OCT_WeldDepth}。然而,OCT 图像存在噪声强、对比度低、边界模糊等特点,使得锁孔区域的稳定提取仍是亟待解决的关键问题。

\subsubsection{OCT图像噪声特性及处理方法研究现状}
\noindent OCT 成像基于低相干干涉,散斑噪声来源于相干叠加与多次散射等因素,常表现为随机颗粒纹理并伴随局部对比度下降,其统计特性与组织/材料表面微结构密切相关\upcite{ZengHaitao_OCT_Speckle}。为提升后续结构识别与测量的稳定性,国内外研究提出了多种 OCT 噪声抑制与质量增强方法,主要包括:\textbf{(1)传统滤波与变换域方法},如中值/均值滤波、各向异性扩散、BM3D 及小波/非局部均值等;\textbf{(2)硬件或多帧复合策略},通过多角度/多次采样实现散斑平均以提升信噪比;\textbf{(3)深度学习去噪方法},利用卷积网络或自编码结构学习噪声分布并恢复结构细节。上述方法为改善可视化质量提供了有效途径,但在锁孔这类\textbf{弱边界、细长深孔、形态快速变化}的目标上,单纯追求去噪往往可能带来边界平滑、细节损失与几何偏移;同时,“先去噪再分割/测量”的两阶段流程增加了系统复杂度与推理延迟,不利于在线闭环控制。
\par\noindent 因此,本文不将“去噪”作为独立处理步骤,而是将散斑与干扰视为成像条件的一部分,探索\textbf{端到端的抗噪语义分割}:通过在分割网络内部显式增强全局上下文建模与局部边界细节表达,使模型在噪声干扰下仍能稳定学习锁孔的语义结构与轮廓,从流程上减少对额外预处理的依赖。

\subsubsection{图像语义分割算法研究现状}
\noindent 语义分割方法经历了从全卷积网络(FCN)到编码器-解码器结构(U-Net 及其变体),再到多尺度上下文建模(DeepLab 系列空洞卷积与 ASPP)等阶段;近年来,Transformer 及注意力机制被引入分割任务,通过自注意力或混合架构增强长程依赖建模能力,在医学影像与复杂场景分割中表现突出\upcite{Long2015FCN,Ronneberger2015UNet,Chen2018DeepLab,TransUNet2021}。在 OCT 相关任务中,研究者多采用 U-Net/DeepLab 类结构进行层结构或病灶区域分割,也有工作引入注意力机制以提升边界与小目标刻画能力。
\par\noindent 然而,激光焊接 OCT 锁孔分割与常规医学 OCT 分割在数据分布与目标形态上存在差异:其一,散斑噪声与工况扰动更强,导致特征不稳定;其二,锁孔区域往往呈现\textbf{细长结构}且边界灰度对比度低,易出现断裂与粘连;其三,在线应用对推理速度与稳定性提出更高要求。上述特点使得现有通用分割网络直接迁移时可能面临全局结构判断不足与局部边界细节丢失等问题。基于此,本文以 DeepLabV3+ 为基线,分别从\textbf{全局上下文}与\textbf{局部细节}两方面进行针对性增强,为后续章节提出的 TR 模块与 SAE 模块提供动机与理论依据。

\subsection{本文主要研究内容}

针对激光焊接 OCT 图像中存在的散斑噪声干扰严重、锁孔目标细小且边界模糊等问题,本文以实现高精度、鲁棒的锁孔语义分割为目标,开展了一系列研究工作。本文的主要研究内容如下:

\begin{enumerate}[label=\arabic*., leftmargin=*, itemsep=0.4\baselineskip, topsep=0.2\baselineskip]
    \item \textbf{构建了面向激光焊接锁孔检测的 OCT 图像语义分割数据集。} 针对现有开源数据集缺乏此类工业场景数据的问题,本文收集了真实的 304 不锈钢激光焊接 OCT 成像数据,制定了统一的标注规范,完成了像素级的精细标注。数据集涵盖了不同焊接工艺参数下的多种锁孔形态,为模型训练与评估提供了坚实的数据基础。

    \item \textbf{提出了一种基于改进 DeepLabV3+ 的 OCT 图像语义分割网络。}
    
    \hspace{2em}针对 DeepLabV3+ 模型在处理 OCT 图像时存在的全局上下文信息利用不足和局部细节丢失问题,本文在编码器-解码器架构的基础上进行了双重改进:
    \begin{enumerate}[label=(\arabic*), leftmargin=3em, itemsep=0.2\baselineskip, topsep=0.2\baselineskip]
        \item 在编码器末端引入\textbf{TR(Transformer Routing)模块}。借鉴 Transformer 全局建模思想,在编码器高层特征上引入基于路由稀疏注意力的全局上下文增强,以提升长程依赖建模能力并降低无关区域干扰\upcite{zhu2023biformervisiontransformerbilevel,TransUNet2021}。
        \item 在解码器融合阶段引入\textbf{SAE(Spatial Attention Enhancement)模块}。在高低层特征融合后引入空间与通道注意力协同增强,突出边缘与细小结构的特征响应,提升弱边界条件下的分割精细度\upcite{hou2021coordinateattentionefficientmobile,hu2019squeezeandexcitationnetworks}。
    \end{enumerate}

    \item \textbf{设计了适应类别不平衡的混合损失函数与训练策略。} 针对 OCT 图像中背景区域广大而锁孔目标区域较小(前景背景比例失衡)的问题,采用交叉熵损失(Cross Entropy Loss)与 Dice 损失(Dice Loss)相结合的混合损失函数,平衡了模型对不同类别的关注度,进一步提升了分割精度(mIoU)。

    \item \textbf{进行了系统的实验验证与对比分析。} 在自建数据集上对所提方法进行了全面的实验评估。实验结果表明,改进后的模型在 mIoU、Dice 系数等关键指标上均优于基线 DeepLabV3+ 及 U-Net、TransUNet 等主流分割网络。通过消融实验验证了 TR 模块与 SAE 模块的有效性,并对分割结果进行了可视化分析,证明了本文方法在复杂工况下的鲁棒性与优越性。
\end{enumerate}

\subsection{论文组织结构}

本文共分为五章,各章节的具体安排如下:

\textbf{第一章:绪论。} 阐述了课题的研究背景及意义,分析了激光焊接过程监测面临的挑战及 OCT 技术的应用潜力。综述了国内外在激光焊接监测、OCT 图像去噪及语义分割算法方面的研究现状,指出了现有研究的不足。最后明确了本文的研究目标、主要研究内容及章节安排。

\textbf{第二章:相关理论与技术基础。} (待补充:介绍 OCT 成像原理、深度学习基础、卷积神经网络、语义分割常用评价指标等。)

\textbf{第三章:基于改进 DeepLabV3+ 的 OCT 图像语义分割方法。} 详细阐述本文提出的改进网络架构。重点介绍 DeepLabV3+ 基线模型、TR 全局注意力模块、SAE 空间细节增强模块的设计原理与实现细节,以及混合损失函数的定义。

\textbf{第四章:实验结果与分析。} 介绍实验数据集的构建、实验环境与参数设置。展示本文方法与主流对比方法的定量评估结果(如 mIoU、PA 等)及定性可视化效果。通过消融实验深入分析各改进模块对模型性能的贡献,并对实验结果进行讨论。

\textbf{第五章:总结与展望。} 总结全文的研究工作与创新点,客观分析现有方法的局限性,并对未来的研究方向(如轻量化部署、多任务协同等)进行展望。
