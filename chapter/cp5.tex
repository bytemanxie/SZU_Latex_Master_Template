\clearpage

\section{总结与展望}

本章对全文的研究工作进行总结,客观分析现有方法的局限性,并对未来的研究方向进行展望。

\subsection{论文总结}

\subsubsection{研究背景与意义回顾}

激光焊接作为现代制造业中的重要工艺,其过程监测对保证焊接质量和生产效率具有重要意义。锁孔作为激光焊接过程中的关键特征,其形态和位置直接影响焊接质量。传统的监测方法主要依赖人工经验,难以实现实时、准确的监测。

光学相干层析成像(OCT)技术作为一种高分辨率、非侵入性的层析成像技术,在激光焊接过程监测中具有重要应用潜力。OCT 技术能够实时获取焊接过程的层析图像,为锁孔检测提供了新的技术手段。然而,OCT 图像存在散斑噪声、对比度低、边界模糊等特点,传统的图像处理方法难以准确分割锁孔区域。

深度学习技术的发展为 OCT 图像语义分割提供了新的解决方案。语义分割作为计算机视觉领域的重要任务,能够实现像素级的分类,为锁孔检测提供了精确的技术手段。然而,现有的语义分割方法在处理 OCT 图像时仍面临挑战:全局上下文信息利用不足、局部细节信息丢失、类别不平衡问题等。

\subsubsection{主要研究工作总结}

针对上述问题,本文围绕基于改进 DeepLabV3+ 的 OCT 图像语义分割方法展开研究,主要工作包括:

\textbf{(1)构建了面向激光焊接锁孔检测的 OCT 图像语义分割数据集}:针对现有开源数据集缺乏激光焊接 OCT 图像数据的问题,本文收集了真实的 304 不锈钢激光焊接 OCT 成像数据,并完成了像素级的精细标注。数据集共包含 1,140 张图像,按照 8:2 的比例划分为训练集和测试集,涵盖了不同焊接工艺参数下的多种锁孔形态,能够较好地反映实际应用场景的多样性。

\textbf{(2)提出了一种基于改进 DeepLabV3+ 的 OCT 图像语义分割网络}:本文以 DeepLabV3+ 为基础框架,引入 TR(Transformer Routing)全局注意力模块和 SAE(Spatial Attention Enhancement)局部注意力模块,形成双重注意力机制。TR 模块通过双层路由注意力机制,增强全局上下文建模能力,建立长距离依赖关系;SAE 模块通过坐标注意力和通道注意力的协同增强,提升局部细节和边界信息的表征能力。两者结合,使得网络能够同时处理全局和局部信息,从而取得更好的分割效果。

\textbf{(3)设计了适应类别不平衡的混合损失函数与训练策略}:针对 OCT 图像中类别不平衡的问题,本文设计了混合损失函数,结合 Binary Cross-Entropy(BCE)损失和 Dice 损失,权重系数分别为 $\lambda_1 = 2.0$ 和 $\lambda_2 = 2.0$。混合损失函数能够同时关注像素级分类准确性和区域重叠度,有效缓解类别不平衡问题。同时,本文采用 Adam 优化器和多项式衰减学习率策略,确保训练过程的稳定性。

\textbf{(4)进行了系统的实验验证与对比分析}:本文在相同实验设置下与基线模型及多种主流语义分割方法进行对比,包括 UNet、UNet++、ResUNet、TransUNet 等。实验结果表明,本文方法在分割性能、目标类别分割和边界精度等方面均取得显著提升。通过消融实验验证了各模块的有效性和协同效应,通过可视化分析验证了方法在不同场景下的鲁棒性,通过效率分析验证了方法的实用性。

\subsubsection{主要创新点}

本文的主要创新点包括:

\textbf{(1)双重注意力机制(全局+局部)}:本文提出了一种双重注意力机制,结合 TR 模块的全局上下文增强和 SAE 模块的局部细节增强。TR 模块通过双层路由注意力机制,建立长距离依赖关系,增强全局上下文建模能力;SAE 模块通过坐标注意力和通道注意力的协同增强,提升局部细节和边界信息的表征能力。两者结合,使得网络能够同时处理全局和局部信息,从而取得更好的分割效果。

\textbf{(2)端到端的抗噪分割}:本文方法采用端到端的训练方式,能够直接从原始 OCT 图像学习到分割结果,无需额外的预处理步骤。通过双重注意力机制,网络能够有效抑制散斑噪声的影响,同时保留重要的边界信息,实现了端到端的抗噪分割。

\textbf{(3)高效计算设计(TopK 路由、深度可分离卷积)}:本文方法在保持高性能的同时,注重计算效率的优化。TR 模块采用 TopK 路由策略,仅对最相关的 K 个窗口进行注意力计算,降低了计算复杂度;SAE 模块采用深度可分离卷积,减少了参数量和计算量。这些设计使得本文方法在保持高性能的同时,具有合理的计算效率。

\subsubsection{实验成果}

实验结果表明,本文方法在 OCT 图像语义分割任务中取得了良好的性能:

\textbf{(1)分割精度显著提升}:mIoU 达到 0.911,相比基线模型提升 7.1\%,在所有对比方法中排名第一。

\textbf{(2)目标类别分割大幅改善}:目标类别 IoU 达到 0.826,相比基线模型提升 16.3\%,相比 UNet 提升 33.2\%,有效解决了类别不平衡问题。

\textbf{(3)边界精度显著提升}:边界精度 HD95 达到 10.68,相比基线模型降低 12.6\%,边界分割更加精确。

\textbf{(4)计算效率优化}:相比 TransUNet,内存占用减少 53.3\%,推理时间减少 39.1\%,FPS 达到 10.5,满足实时应用需求。

综合以上实验结果,本文方法在 OCT 图像语义分割任务中取得了良好的性能,验证了所提方法的有效性。

\subsection{方法局限性分析}

尽管本文方法在 OCT 图像语义分割任务中取得了良好的性能,但仍存在一些局限性,需要在未来的研究中进一步改进。

\subsubsection{参数量与计算复杂度}

本文方法的参数量为 420 MB,相比基线模型(95 MB)增加 342.1\%,推理时间为 95 ms,相比基线模型(38 ms)增加 150.0\%。参数量和计算复杂度的增加主要来源于 TR 模块和 SAE 模块的引入。

\textbf{(1)参数量增加}:TR 模块的参数量为 301 MB,SAE 模块的参数量为 25 MB,两者结合使得总参数量达到 420 MB。虽然参数量增加较大,但性能提升显著(mIoU 提升 7.1\%),参数效率合理。

\textbf{(2)计算复杂度增加}:TR 模块的双层路由注意力机制需要计算窗口间相似度和 TopK 选择,SAE 模块的坐标注意力和通道注意力需要额外的特征聚合和权重计算,这些都会增加计算复杂度。虽然推理时间增加 150.0\%,但 FPS 仍达到 10.5,满足实时应用需求。

\textbf{(3)对计算资源要求较高}:本文方法的内存占用为 9107 MB,需要较大的 GPU 显存。虽然在现代 GPU(如 24 GB 显存)上可以运行,但对计算资源的要求较高,限制了其在资源受限环境下的应用。

\subsubsection{数据集局限性}

本文方法的数据集存在以下局限性:

\textbf{(1)数据集规模相对较小}:训练集包含 912 张图像,测试集包含 228 张图像,数据集规模相对较小。虽然通过数据增强策略能够增加训练数据的多样性,但数据集规模仍然有限,可能影响模型的泛化能力。

\textbf{(2)主要针对 304 不锈钢激光焊接场景}:数据集主要针对 304 不锈钢激光焊接场景,对其他材料(如铝合金、钛合金等)或不同工艺参数下的泛化能力有待验证。不同材料和工艺参数下的锁孔形态可能存在差异,需要进一步验证方法的泛化能力。

\textbf{(3)对其他材料或工艺的泛化能力有待验证}:本文方法在 304 不锈钢激光焊接场景下取得了良好的性能,但对其他材料或工艺的泛化能力尚未进行充分验证。未来需要通过收集更多不同材料和工艺的数据,验证方法的泛化能力。

\subsubsection{方法局限性}

本文方法在以下方面仍存在局限性:

\textbf{(1)极端噪声条件下的误分割}:在极端噪声条件下(如散斑噪声非常严重、图像质量极差),本文方法可能出现误分割或漏分割问题。原因可能是噪声干扰过大,导致特征提取困难,即使通过注意力机制增强,仍难以准确识别目标区域。

\textbf{(2)形态异常目标的处理能力有限}:对于形态异常的目标(如锁孔形状极其不规则、存在多个锁孔重叠等),本文方法可能出现分割不完整或误分割问题。原因可能是训练数据中此类样本较少,模型未能充分学习此类形态的特征。

\textbf{(3)实时性优化空间}:虽然本文方法的 FPS 达到 10.5,满足实时应用需求,但相比基线模型(26.3 FPS)仍有较大差距。如果对实时性要求更高,需要进一步优化计算效率,如采用模型压缩、架构优化等技术。

\textbf{(4)边界极度模糊情况下的定位不准确}:对于边界极度模糊的情况(如锁孔边缘几乎不可见),本文方法虽然相比基线方法有所提升,但仍可能出现边界定位不准确的问题。原因可能是边界信息本身不足,即使通过注意力机制增强,也难以完全恢复边界信息。

针对上述局限性,未来可以通过扩大数据集规模、增加数据增强策略、优化网络结构、采用模型压缩技术等方式进一步改进。

\subsection{研究展望}

基于本文的研究成果和局限性分析,未来的研究工作可以从以下几个方面展开:

\subsubsection{轻量化部署}

\textbf{(1)模型压缩技术}:采用知识蒸馏、剪枝、量化等技术,在保持性能的同时降低参数量和计算复杂度。知识蒸馏通过教师-学生网络结构,将大模型的知识迁移到小模型;剪枝通过移除不重要的连接或通道,减少模型参数;量化通过降低数值精度,减少存储和计算开销。

\textbf{(2)移动端/边缘设备部署}:针对移动端和边缘设备的资源限制,需要进一步优化模型结构,降低内存占用和计算复杂度。可以考虑采用轻量级的注意力机制,如 MobileViT、EfficientNet 等,在保持性能的同时降低资源需求。

\textbf{(3)实时性进一步优化}:通过优化网络结构、采用更高效的注意力机制、利用专用硬件加速等方式,进一步提升推理速度。可以考虑采用 TensorRT、ONNX Runtime 等推理框架,利用 GPU 的并行计算能力加速推理。

\subsubsection{多任务协同}

\textbf{(1)同时进行锁孔分割和熔深估计}:在锁孔分割的基础上,同时进行熔深估计,实现多任务学习。通过共享编码器特征,可以同时学习分割和回归任务,提高模型的效率和泛化能力。

\textbf{(2)多模态信息融合}:结合 OCT 图像、视觉图像和声学信号等多模态信息,通过多模态融合提升分割性能。不同模态的信息具有互补性,融合后能够提供更丰富的特征表示。

\textbf{(3)端到端的多任务学习框架}:设计端到端的多任务学习框架,同时完成锁孔分割、熔深估计、质量评估等多个任务。通过任务间的信息共享和协同学习,可以提高模型的效率和性能。

\subsubsection{数据增强与泛化}

\textbf{(1)扩大数据集规模}:收集更多不同材料、不同工艺参数下的 OCT 图像数据,扩大数据集规模。更大的数据集能够提供更丰富的样本多样性,提高模型的泛化能力。

\textbf{(2)不同材料、不同工艺参数的数据收集}:针对不同材料(如铝合金、钛合金等)和不同工艺参数(如激光功率、焊接速度、离焦量等),收集相应的 OCT 图像数据,验证方法的泛化能力。

\textbf{(3)域适应技术(Domain Adaptation)}:采用域适应技术,将模型从源域(训练数据)适应到目标域(测试数据),提高模型的泛化能力。可以考虑采用对抗训练、特征对齐等方法,减少域间差异。

\textbf{(4)少样本学习(Few-shot Learning)}:针对数据稀缺的场景,采用少样本学习技术,通过少量样本快速适应新任务。可以考虑采用元学习、迁移学习等方法,提高模型的快速适应能力。

\subsubsection{方法改进}

\textbf{(1)更高效的注意力机制设计}:设计更高效的注意力机制,在保持性能的同时降低计算复杂度。可以考虑采用局部注意力、稀疏注意力等方法,减少注意力计算的开销。

\textbf{(2)自适应损失函数权重调整}:设计自适应损失函数权重调整策略,根据训练过程中的性能变化动态调整损失函数权重,提高训练效率和模型性能。

\textbf{(3)在线学习与增量学习}:针对实际应用中的新数据,采用在线学习和增量学习技术,使模型能够持续学习和适应新数据,而无需重新训练整个模型。

\subsubsection{应用拓展}

\textbf{(1)其他工业场景的应用}:将本文方法拓展到其他工业场景,如缺陷检测、质量评估等。OCT 技术在工业检测领域具有广泛应用,本文方法可以进一步拓展应用范围。

\textbf{(2)医学图像分割的应用}:将本文方法应用到医学图像分割任务,如视网膜分割、血管分割等。OCT 技术在医学领域具有重要应用,本文方法可以进一步拓展应用领域。

\textbf{(3)与其他监测技术的融合}:结合其他监测技术,如视觉监测、声学监测等,通过多模态融合提升监测效果。不同监测技术具有互补性,融合后能够提供更全面的监测信息。

综合以上展望,未来的研究工作可以从轻量化部署、多任务协同、数据增强与泛化、方法改进和应用拓展等多个方面展开,进一步提升方法的性能、效率和实用性。